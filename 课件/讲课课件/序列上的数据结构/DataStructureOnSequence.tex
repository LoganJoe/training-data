\documentclass[UTF8]{beamer}

\usepackage{ctex}
\usepackage{setspace}
\usepackage{indentfirst}
\usepackage{ulem}
\usepackage{wasysym}
\usepackage{graphicx}

\usetheme{Szeged}
%\usecolortheme{seahorse}

\setlength{\parindent}{1.5em}
\setlength\parskip{.3\baselineskip}

\title{序列上的数据结构}

\author{h10}

\begin{document}

	\begin{frame}

		\maketitle

	\end{frame}

	\section{数据结构简介}

		\subsection{数据结构简介}

			\begin{frame}{简介}

			在计算机科学中,数据结构是计算机存储、组织数据的方式,数据结构是指相互之间存在一种或多种特定关系的数据元素的集合

			通常情况下,精心选择的数据结构可以带来更高的运行或者存储效率,数据结构往往同高效的检索算法和索引技术有关

			\end{frame}

			\begin{frame}{定义}

			数据结构是指相互之间存在着一种或多种关系的数据元素的集合和该集合中数据元素之间的关系组成,记为:

			$$
			Data_Structure=(D,R)
			$$

			其中 $D$ 是数据元素的集合,$R$ 是该集合中所有元素之间的关系的有限集合

			\end{frame}

			\begin{frame}{常见数据结构}

			数组(Array)

			堆栈(Stack)

			队列(Queue)

			链表(Linked List)

			树(Tree)

			图(Graph)

			堆(Heap)

			散列表(Hash)

			\end{frame}

		\subsection{数据结构分类}

			\begin{frame}{逻辑结构分类}

			数据的逻辑结构是从具体问题抽象出来的数学模型,是描述数据元素及其关系的数学特性的,逻辑结构是在计算机存储中的映像,形式地定义为 $(K,R)$,其中,$K$ 是数据元素的有限集,$R$ 是 $K$ 上的关系的有限集

			根据数据元素间关系的特性,通常有下列四类基本的结构:

			1.集合结构:数据结构中的元素之间除了同属一个集合的相互关系外,别无其他关系

			2.线性结构:数据结构中的元素存在一对一的相互关系,其中线性表是最简单、最基本、也是最常用的一种线性结构

			3.树形结构:数据结构中的元素存在一对多的相互关系

			4.图形结构:数据结构中的元素存在多对多的相互关系

			\end{frame}

			\begin{frame}{物理(存储)结构分类}

			数据结构在计算机中的表示称为数据的物理(存储)结构,它包括数据元素的表示和关系的表示

			数据元素之间的关系有两种不同的表示方法:顺序映象和非顺序映象,并由此得到两种不同的存储结构:顺序存储结构和链式存储结构

			顺序存储方法:它是把逻辑上相邻的结点存储在物理位置相邻的存储单元里,结点间的逻辑关系由存储单元的邻接关系来体现,由此得到的存储表示称为顺序存储结构

			链接存储方法:它不要求逻辑上相邻的结点在物理位置上亦相邻,结点间的逻辑关系是由附加的指针字段表示的,由此得到的存储表示称为链式存储结构,链式存储结构通常借助于程序设计语言中的指针类型来实现

			\end{frame}

			\begin{frame}{物理(存储)结构分类}

			处顺序存储结构和链式存储结构外还有其它存储结构:

			索引存储方法:除建立存储结点信息外,还建立附加的索引表来标识结点的地址

			散列存储方法:就是根据结点的关键字直接计算出该结点的存储地址

			\end{frame}

	\section{OI基础数据结构}

		\subsection{$O(n)$数据结构}

			\begin{frame}{数组}

			给定一个长度为 $n$ 的数组

			$m$ 次询问,每次询问数组一段区间内所有数字的和

			\end{frame}

			\begin{frame}{链表}

			给出 $n$ 个不同元素的一种排列

			$m$ 次询问,每次询问一个元素的前驱或后继

			\end{frame}

			\begin{frame}{栈 \& 队列}

			DFS \& BFS

			\end{frame}

			\begin{frame}{单调栈 \& 单调队列}

			多用于可以贪心的题目

			\end{frame}

			\begin{frame}{hash}

			给出 $n$ 个名字不同的人的考试成绩

			$m$ 次询问,每次询问一人的成绩

			\end{frame}

		\subsection{$O(n\log(n))$数据结构}

			\begin{frame}{RMQ}

			给定一个长度为 $n$ 的数组

			$m$ 次询问,每次询问数组一段区间内所有数字的最大值或最小值

			\end{frame}

			\begin{frame}{堆}

			优先队列

			\end{frame}

			\begin{frame}{可合并堆}

			给定 $n$ 个集合,开始都为空,$m$ 次操作,每次操作为如下之一

			1. 向某个集合插入一个数字

			2. 从某个集合中取出其中最大的数字并输出

			3. 把两个集合合并为一个

			举例:斜堆,左偏树,随机堆

			\href{https://blog.csdn.net/jacajava/article/details/44680951}{\emph{\underline{详见此处}}}

			\end{frame}

			\begin{frame}{斐波那契堆}

			给定 $n$ 个集合,开始都为空,$m$ 次操作,每次操作为如下之一

			1. 向某个集合插入一个数字

			2. 从某个集合中取出其中最大的数字并输出

			3. 把两个集合合并为一个

			4. 修改某个集合中的某个数的值

			\href{https://www.cnblogs.com/skywang12345/p/3659060.html}{\emph{\underline{详见此处}}}

			\end{frame}

			\begin{frame}{树状数组 \& 线段树}

			给定一个长度为 $n$ 的数组

			$m$ 次操作,每次操作修改某个数字的数值或询问数组一段区间内所有数字的和

			\end{frame}

			\begin{frame}{旋转式平衡树}

			给出 $n$ 个元素的一种排列

			$m$ 次操作,每次操作为如下之一

			在某个位置加入一个新元素

			在某个位置删去一个元素

			将某一段区间的所有元素移至另一位置

			举例:Splay,旋转式treap(注意treap可以线性构造)

			\end{frame}

			\begin{frame}{主席树 \& 非转式treap} 

			主席树:区间第 $k$ 大

			非转式treap:一种可以持久化的平衡树

			一个问题:在可离线的情况下,有什么方案可以替代可持久化

			\end{frame}

			\begin{frame}{替罪羊思想}

			暴力美学...

			不平衡就打扁系列

			与非转式treap一样,也是一种非转式平衡树

			\end{frame}

		\subsection{$O(n\log^2(n))$数据结构}

			\begin{frame}{树套树}

			并不想多说什么...

			\end{frame}

			\begin{frame}{树链剖分}

			树上数据结构问题强行下树!

			类似的,prufer编码则可以使树上计数问题强行下树

			\end{frame}

		\subsection{$O(n\sqrt n)$数据结构}

			\begin{frame}{分块}

			暴力美学...

			\end{frame}

			\begin{frame}{莫队}

			经典例题:\href{http://hzwer.com/2782.html}{\emph{\underline{小z的袜子}}}

			\end{frame}

			\begin{frame}{KD树}

			给出 $n$ 个平面上的点

			$m$ 次询问,每次询问离某个位置最近的 $k$ 个点是那些 $k \le 10$

			\end{frame}

	\section{例题}

		\subsection{简单题}

			\begin{frame}{bzoj2827}

			平面上有 $n$ 只鸟,每只都有自己的初始位置与威武值

			接下来 $t$ 秒间,每秒都会有一只鸟改变自己的位置

			定义一只鸟在某一刻的士气值为此刻与它站在同一位置的所有鸟中最大的威武值,不包括自己

			定义一只鸟在某一刻的团结值为此刻与它站在同一位置的鸟的个数,不包括自己

			求每只鸟的士气值与团结值的历史最大值

			\end{frame}

			\begin{frame}{bzoj2827}

			平衡树裸题,写个哈希表来存坐标

			我们需要这样一种数据结构,它需要支持插入,删除,集合chkmax,维护集合大小,集合max

			随便找种平衡树咯

			\end{frame}

			\begin{frame}{带单点修改的区间第k大}

			题面看标题,懒得写了

			请尝试提出多种解题方案,特别请尝试提出一个空间复杂度为 $O(n\log(n))$ 的算法

			\end{frame}

			\begin{frame}{带单点修改的区间第k大}

			替罪羊树或非转式Treap套值域线段树,时间 $O(n\log^2(n))$,空间 $O(n\log^2(n))$

			值域线段树套线段树,时间 $O(n\log^2(n))$,空间 $O(n\log^2(n))$
                        
			值域线段树套平衡树,时间 $O(n\log^2(n))$,空间 $O(n\log(n))$

			\href{http://vfleaking.blog.163.com/blog/static/1748076342013123659818/}{\emph{\underline{各种算法的效率}}}

			\end{frame}

			\begin{frame}{bzoj3653}

			给出一棵 $n$ 个点的树,以及 $m$ 个询问,每个询问形式如下:

			给定 $p,k$,问有多少个三元组 $(p,x,y)$,满足

			1. $p$ 和 $x$ 都是 $y$ 的祖先

			2. $p$ 和 $x$ 的距离小于等于 $k$

			$n,m \le 10^5$

			\end{frame}

			\begin{frame}{bzoj3653}

			分两种情况

			1. $x$ 是 $p$ 的祖先:非常好处理

			2. $p$ 是 $x$ 的祖先或 $x=p$:

			此时合法的 $x$ 的深度区间为 $[dep(p),dep[p]+k]$,满足条件的 $x$ 对答案有 $sz[x]-1$ 的贡献

			按 DFS 序建立主席树即可

			\end{frame}

		\subsection{不简单题}

			\begin{frame}{bzoj4569}

			有一个长度为 $n$ 的没有前导零的十进制数,用 $s$ 表示,有 $m$ 个限制条件,每个条件形如 $(l1,r1,l2,r2)$,表示 $s[l1:r1]=s[l2:r2]$

			现在给出这些限制条件,问有多少个数满足条件

			$n,m \le 10^5$

			\end{frame}

			\begin{frame}{bzoj4569}

			考虑类似ST表的方法,我们把这个区间划分成前 $2^k$ 位和后 $2^k$ 位,那么就变成了这两端 $2^k$ 位分别对应相同

			我们开 $\log(n)$ 个并查集,第 $k$ 个记录对应的第 $k$ 层的相同性

			处理完所有条件之后,我们从上往下把相同性结果推到下一层去,就可以在总时间 $O((n+m)\log(n)\alpha(n))$ 的复杂度内得到最后的并查集

			\end{frame}

			\begin{frame}{uoj164}

			给定一个长度为 $n$ 的数组 $A$,并定义一个它的辅助数组 $B$,初始 $B$ 与 $A$ 完全相同

			接下来有 $m$ 次操作,每次操作为如下之一($x$ 可为负数)

			1. 给定 $l,r,x$,对于所有的 $i \in [l,r]$,把 $A_i$ 加上 $x$

			2. 给定 $l,r,x$,对于所有的 $i \in [l,r]$,把 $A_i$ 变成 $x$

			3. 给定 $l,r,x$,对于所有的 $i \in [l,r]$,把 $A_i$ 变成 $max(A_i+x,0)$

			4. 给定 $i$,询问 $A_i$ 的数值

			5. 给定 $i$,询问 $B_i$ 的数值

			在每一次操作后对 $B$ 进行一次更新,让 $B_i=max(A_i,B_i)$

			\end{frame}

			\begin{frame}{uoj164}

			首先把操作一般化,所有操作都可以用函数 $x=f_{a,b}(x)$ 表示,意义为对 $x$ 进行 $x=max(x+a,b)$ 的操作

			考虑操作的合并有 $f_{c,d}(f_{a,b}(x))=f_{a+c,max(b+c,d)}(x)$

			由于要求历史最大值,还得维护历史最大标记

			如果对 $x$ 先后进行 $k$ 次操作,第 $i$ 次操作为 $f_{a_i,b_i}(x)$,那么对于 $x$ 的历史最大值 $g(x)$ 有 $g(x)=max_{i=0}^k f_{a_i,b_i}(x)$

			由于 $g(x)$ 为多个 $f$ 函数取最大值,画画图就可以发现 $g(x)$ 也可以写成 $f_{a,b}(x)$ 的形式,也就是说 $g(x)$ 也是可以合并的

			既然两个询问都是可以合并的,线段树一下就好了

			\end{frame}

			\begin{frame}{ICPC-Camp 2016 Day1 F}

			对于两个自然数 $a$ 与 $b$,重新定义 $a \le b$ 为 $ a|b = b$ 其中的 $|$ 表示按位或

			给定一个长度为 $n$ 的全由自然数组成的数组 $a$,求它的非空不下降子序列个数

			$n \le 10^5, a_i \le 2^{16}$

			\end{frame}

			\begin{frame}{ICPC-Camp 2016 Day1 F}

			设 $dp[i]$ 为以 $a[i]$ 结尾的不下降子序列个数

			假设当前考虑到了第 $i$ 个数字,记 $cnt[mask1][mask2]$ 为 $a[1]$ 至 $a[i-1]$ 中前 $8$ 位正好是 $mask1$,后 $8$ 位小于等于 $mask2$ 的数字的个数

			暴力搜索所有可行的 $mask1$,即可得出 $dp[i]$

			然后需要用 $dp[i]$ 更新 $cnt$,由于 $mask1$ 是已经确定了的,所以只要遍历所有 $mask2$ 即可

			时间复杂度:$O(n*2^8)$

			\end{frame}

	\section{不常见数据结构}

		\subsection{也就是NOI应该不会考的东西}

			\begin{frame}{列表}

			1. 二进制分组

			2. 吉司机线段树

			3. 划分树

			这里我偷个懒,直接放以前课件了

			\end{frame}


\end{document}

complex
32叉线段树
舞蹈链

what I need to learn
哈夫曼树

