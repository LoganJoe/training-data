\documentclass{article}

\usepackage{ctex}
\usepackage{amsmath}
\usepackage{amssymb}
\usepackage{graphicx}
\usepackage{color}
\usepackage{textcomp}
\usepackage{fancyhdr}
\usepackage{lastpage}
\usepackage{fontspec}
\usepackage{multirow}
\usepackage[colorlinks,linkcolor=black]{hyperref}

%\setmainfont[BoldFont=Source Sans Pro]{Source Sans Pro}
%\setmonofont{Consolas}
%\setCJKmainfont[BoldFont=Microsoft YaHei Semibold]{Microsoft YaHei Semilight}

\addtolength{\hoffset}{-1.2cm}
\addtolength{\marginparwidth}{-1.2cm}
\addtolength{\textwidth}{2.4cm}

\addtolength{\voffset}{-1cm}
\addtolength{\textheight}{2.0cm}

\pagestyle{fancy}
\renewcommand{\headrulewidth}{0.1pt}
\renewcommand{\footrulewidth}{0.1pt}
\lhead{by superguymj}
\rhead{constitution}
\cfoot{\thepage /\pageref{LastPage}}

\title{董先生的休闲}
\author{superguymj}
\date{March 2018}

\begin{document}

\section*{董先生的休闲方案(constitution.c/cpp/pas)}

%\fancyhead[R]{constitution}

\subsection*{题目描述}

桌上摆着$n$份董先生的休闲方案。
现在董先生需要将这些休闲方案按高明程度\textbf{从小到大}依序提出。

所有方案的高明程度是一个$1 \dots n$的排列,但是高明程度都写在方案的第二页,董先生一开始并不知道每一份方案的高明程度,因此他要进行评估(也就是翻到第二页查看高明程度)。

每一秒,若董先生已经知道下一份要提出的休闲方案是哪一份了,便会将其提出;否则董先生会在\textbf{仍未进行评估}的休闲方案中\textbf{等概率}地取出一份进行评估,若恰好是下一份要提出的休闲方案,董先生便会将其直接提出,否则董先生会将其放回桌上(注意这些操作都在一秒内完成)。

但是董先生急着去打游戏,所以他将这个任务交给了你,并希望你能告诉他耗时的\textbf{数学期望}。

\subsection*{输入格式(constitution.in)}

一行三个正整数$p,k,n$,$p,k$的意义见输出格式。

\subsection*{输出格式(constitution.out)}

一行一个整数,表示在\textbf{模$p^k$意义下}的期望耗时。\textbf{数据保证有解},详见数据范围。

\subsection*{样例一输入}

2333 1 3

\subsection*{样例一输出}

393

\subsection*{样例二输入}

3 2 7

\subsection*{样例二输出}

8

\subsection*{样例三输入}

3 2 8

\subsection*{样例三输出}

2

\subsection*{样例四输入}

7 6 719102

\subsection*{样例四输出}

8533

\subsection*{样例解释}

共有1--2--3、1--3--2--3、2--1--2--3、3--1--2--3、2--3--1--2--3、3--2--1--2--3六种可能的顺序,期望耗时$\frac{25}{6} \equiv 393 \pmod{2333}$。

其中$x$代表高明程度为$x$的方案,若$x$出现了两次,则第一次表示评估了$x$,第二次表示提交了$x$。

\subsection*{数据范围与提示}

记期望耗时为$\frac{u}{v}$,其中$u,v$互素。

若存在整数$v^{-1}$,满足$vv^{-1} \equiv 1 \pmod{p^k}$,则模$p^k$意义下的答案即为$uv^{-1} \bmod p^k$。

对于100\%的数据,保证存在整数$v^{-1}$,$p$为奇素数,且$p \leq 10^5$,$np^k \leq 10^{18}$。

下表给出了部分数据点的详细数据范围。
\begin{center}
    \begin{tabular}{|c|c|c|c|} \hline
        数据点编号 & $n \leq$ & $np^k \leq$ & 特殊性质 \\ \hline
        1 & $10$ & \multirow{6}{*}{$10^{12}$} & \multirow{4}{*}{$n<p$且$k=1$} \\ \cline{1-2}
        2 & $15$ & & \\ \cline{1-2}
        3 & $2000$ & & \\ \cline{1-2}
        4 & $10^5$ & & \\ \cline{1-2} \cline{4-4}
        5 & $1000$ & & $k=2$ \\ \cline{1-2} \cline{4-4}
        6 & $2 \times 10^5$ & & \multirow{5}{*}{无} \\ \cline{1-3}
        7 & \multicolumn{2}{|c|}{\multirow{4}{*}{无}} & \\ \cline{1-1}
        8 & \multicolumn{2}{|c|}{} & \\ \cline{1-1}
        9 & \multicolumn{2}{|c|}{} & \\ \cline{1-1}
        10 & \multicolumn{2}{|c|}{} & \\ \hline
    \end{tabular}
\end{center}

\end{document}