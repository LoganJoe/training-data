\documentclass{article}

\usepackage{ctex}
\usepackage{amsmath}
\usepackage{amssymb}
%\usepackage{graphicx}
%\usepackage{color}
%\usepackage{textcomp}
\usepackage{fancyhdr}
\usepackage{lastpage}
%\usepackage{fontspec}
%\usepackage{multirow}
%\usepackage[colorlinks,linkcolor=black]{hyperref}

%\setmainfont[BoldFont=Source Sans Pro]{Source Sans Pro}
%\setmonofont{Consolas}
%\setCJKmainfont[BoldFont=Microsoft YaHei Semibold]{Microsoft YaHei Semilight}

\addtolength{\hoffset}{-1.2cm}
\addtolength{\marginparwidth}{-1.2cm}
\addtolength{\textwidth}{2.4cm}

\addtolength{\voffset}{-1cm}
\addtolength{\textheight}{2.0cm}

\pagestyle{fancy}
\renewcommand{\headrulewidth}{0.1pt}
\renewcommand{\footrulewidth}{0.1pt}
\lhead{by superguymj}
\chead{solution}
\rhead{constitution}
\cfoot{\thepage /\pageref{LastPage}}

\title{董先生的休闲}
\author{superguymj}
\date{March 2018}

\begin{document}

\section*{董先生的休闲方案}

\subsection*{解法一}

每个方案只会评估一次,显然评估顺序与$1 \dots n$的排列一一对应。

$x$只算一次当且仅当排列中比$x$小的数都在$x$的前面。

枚举排列即可做到$\mathcal{O}(n!n)$的时间复杂度。

期望得分10\%。

\subsection*{解法二}

显然可以状态压缩。

$f_{i,S}$表示已经提出了前$i$个方案,目前已知方案为$S$,仍需耗时的期望。

转移显然。

时间复杂度$\mathcal{O}(n2^n)$。

期望得分20\%。

%\subsection*{解法二(20\%)}

%$\mathcal{O}(n^3)$。
%做法:
%首先,我们发现这题可以不管每次具体选了什么数,只看已经拿到几个数和进行了几次操作,这样就可以搞一个概率dp,f[i][j]表示现在已经翻了i次,数到了第j个数,进行转移。
%然而这里有个问题,就是当前回合翻出来的数,这种转移可能是会把它看成答案的。举个例子:
%f[0][0] 翻到2 -> f[1][0]
%f[1][0] 误认为之前翻过1 -> f[1][1]
%因此,我们必须多开一维k表示当前回合翻过多少个错的数,当前回合选过的数不能再选,然后在转移到下一回合的时候再加进i中。

\subsection*{解法三}

因为期望是线性函数,每个方案对答案的贡献是独立的。

考虑计算有多少个排列,满足排列中比$x$小的数都在$x$的前面。

枚举$x$前比$x$大的数的个数,容易得到
\[
f(x) = \sum_{i=0}^{n-x} \binom{n-x}{i} (x-1+i)! (n-x-i)!
\]

答案即为
\[
2n - \sum_{i=1}^n \frac{f(i)}{n!}
\]

时间复杂度$\mathcal{O}(n^2)$。

期望得分30\%。

\subsection*{解法四}

化简$f(x)$,容易得到
\begin{eqnarray*}
f(x)
& = & \sum_{i=0}^{n-x} \frac{(n-x)! (x-1+i)! (n-x-i)!}{i! (n-x-i)!}  \\
& = & (n-x)! (x-1)! \sum_{i=0}^{n-x} \frac{(x-1+i)!}{i!(x-1)!} \\
& = & (n-x)! (x-1)! \sum_{i=0}^{n-x} \binom{x-1+i}{i} \\
& = & (n-x)! (x-1)! \binom{n}{n-x} \\
& = & \frac{(n-x)! (x-1)! n!}{x! (n-x)!} \\
& = & \frac{n!}{x}
\end{eqnarray*}

化简和式使用了平行求和公式。

形象地理解,比$x$大的数都确定后,$x$的位置就已经确定了,所以有$\frac{n!}{x}$种排列。

答案即为
\[
2n-\sum_{i=1}^n \frac{1}{i}
\]

时间复杂度$\mathcal{O}(n)$。

期望得分40\%。

\subsection*{解法五}

考虑优化解法二的状态。

$f_i$表示剩余$i$份方案未评估的期望耗时,显然有
\[
f_i
= \frac{1}{i}(f_{i-1}+1)+(1-\frac{1}{i})(f_{i-1}+2)
= f_{i-1}+2-\frac{1}{i}
\]

答案即为
\[
f_n=2n-\sum_{i=1}^n \frac{1}{i}
\]

时间复杂度$\mathcal{O}(n)$。

期望得分40\%。

\subsection*{解法六}
 
设\[F(n,p,k) \equiv \sum_{i=1}^n \frac{1}{i} \pmod{p^k}\]

\begin{eqnarray*}
F(n,p,k)
& \equiv & \sum_{i=1}^n \frac{[p \nmid i]}{i} + \sum_{i=1}^{\lfloor \frac{n}{p} \rfloor} \frac{1}{ip} \pmod{p^k}\\
& \equiv & \left( \sum_{i=1}^n \frac{[p \nmid i]}{i} \right) + \frac{F(\lfloor \frac{n}{p} \rfloor ,p,k+1)}{p} \pmod{p^k}
\end{eqnarray*}

时间复杂度$\mathcal{O}(n \log_p n)$。

期望得分60\%。

\subsection*{解法七}

设\[f(n,p,k) \equiv \sum_{i=1}^n \frac{[p \nmid i]}{i} \pmod{p^k}\]

则\[f(n,p,k) \equiv \sum_{i=1}^{p-1} \sum_{j=0}^{\lfloor \frac{n}{p} \rfloor -1} \frac{1}{jp+i} + \sum_{i=\lfloor \frac{n}{p} \rfloor p+1}^n \frac{1}{i} \pmod{p^k}\]

其中$\sum_{i=\lfloor \frac{n}{p} \rfloor p+1}^n \frac{1}{i}$可以$\mathcal{O}(p)$求得。

将$\sum_{j=0}^{\lfloor \frac{n}{p} \rfloor -1} \frac{1}{jp+i}$展开有
\[
\sum_{i=1}^{p-1} \sum_{j=0}^{\lfloor \frac{n}{p} \rfloor -1} \frac{1}{jp+i}
\equiv \sum_{i=1}^{p-1} \sum_{j=0}^{\lfloor \frac{n}{p} \rfloor -1} \sum_{x=0}^{\infty} \frac{(-1)^x j^x p^x}{i^{x+1}} \pmod{p^k}
\]

当$x \geq k$时,$p^x \equiv 0 \pmod{p^k}$,因此
\[
\sum_{i=1}^{p-1} \sum_{j=0}^{\lfloor \frac{n}{p} \rfloor -1} \sum_{x=0}^{k-1} \frac{(-1)^x j^x p^x}{i^{x+1}}
\equiv
\sum_{x=0}^{k-1} \sum_{i=1}^{p-1} \frac{(-1)^x p^x}{i^{x+1}} \sum_{j=0}^{\lfloor \frac{n}{p} \rfloor -1} j^x \pmod{p^k}
\]

其中$\sum_{j=0}^{\lfloor \frac{n}{p} \rfloor -1} j^x$可以用伯努利数$\mathcal{O}(k^2)$或矩阵乘法快速幂$\mathcal{O}(k^3 \log n)$预处理,因而上式可以$\mathcal{O}(pk)$求得。

总时间复杂度$\mathcal{O}((pk + k^2) \log_p n)$或$\mathcal{O}((pk+k^3 \log n) \log_p n)$。

期望得分100\%。

\subsection*{数据}

因为有解的点非常少,所以$p,k,n$都不会非常满,拿了四台电脑跑了一个星期终于把最后四组数据造了出来。

\end{document}